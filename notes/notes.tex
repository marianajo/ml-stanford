\documentclass[a4paper,11pt]{article}
\usepackage{preamble}

\title{Machine Learning Course\\\large{Notebook}}
\author{Mariana Jó}
\date{\today}

\begin{document}
  \maketitle
  \thispagestyle{fancy}


  \section{What is Machine Learning?}
  \begin{itemize}
    \item Definitions of ML \\
    \textit{"A computer program is said to learn from experience E with respect to some class of tasks T and performance measure P, if its performance at tasks in T, as measured by P, improves with experience E."}
    \item Kinds of ML algorithms
  \end{itemize}

  \section{Supervised learning}
  Some definitions:
  \begin{itemize}
    \item Supervised Learning: We give the dataset the "right answers"
    \item Regression: predict continous valued output
    \item Classification: discrete valued output
    \item Support Vector Machines: an algorithm that can deal with an infinite number of features
  \end{itemize}
  \section{Unsupervised learning}
  \textit{"Unsupervised learning allows us to approach problems with little or no idea what our results should look like. We can derive structure from data where we don't necessarily know the effect of the variables. With unsupervised learning there is no feedback based on the prediction results."}
  \begin{itemize}
    \item Clustering: groups things that are somehow similar and/or related
    \item Non-clustering: finds structure. Ex.: cocktail party algorithm.
  \end{itemize}

  \section{Model representation}
  \textbf{Notation:}
  \begin{itemize}
    \item $m=$ Number of training examples
    \item $x=$ "input" variable / features
    \item $y=$ "output" variable / "target" variable
    \item $(x,y)=$ one training example
    \item $\theta=$parameter
  \end{itemize}
  About hypothesis $h$:
  \begin{itemize}
    \item Hypothesis $h$: takes the input $x$ and outputs the estimated value $y$, i.e., $h$ maps from $x$'s to $y$'s.
    \item How do we represent $h$? \\
    We choose an initial choice
    \begin{equation}
    h_\theta (x)=\theta_0 +\theta_1 x  ,
    \label{eq:hypothesis}
    \end{equation}
    which is a linear function.
    \item Linear Regression with one variable $=$ Univariate Linar Regression
  \end{itemize}
  Cost function:
  \begin{itemize}
    \item We try to minimize $\theta_0 \theta_1$
    \item So we try to minimize
    \begin{equation}
    J(\theta_0, \theta_1)=\frac{1}{2m}\sum_{i=1}^m \left(h_\theta (x^i) - y^i\right)^2
  \end{equation}
  where $h_\theta (x)$ is given by \ref{eq:hypothesis} and $J(\theta_0, \theta_1)$ is called the \textbf{cost function}. This function is also called \textbf{Squared error function} or \textbf{Mean Squared Error (MSE)}.
  \item The mean is halved $(\frac{1}{2})$ as a convenience for the computation of the gradient descent, as the derivative term of the square function will cancel out the $\frac{1}{2}$ term.
  \end{itemize}

  \section{Parameter Learning}
  Gradient descent
  \begin{itemize}
    \item Given $J(\theta_0, \theta_1)$, we want to minimize it.
    \item Start with some $\theta_0, \theta_1$
    \item Keep changing $\theta_0, \theta_1$ to reduce $J(\theta_0, \theta_1)$
    \item Gradient descent algorithm: \\
    Repeat until convergence:
    \begin{equation}
      \theta_j := \theta_j - \alpha \frac{\partial}{\partial \theta_j}
      J(\theta_0, \theta_1), \quad (\text{for} \ j=0 \ \text{and} \ j=1)
      \label{eq:gradient_descent}
    \end{equation}
    where $\alpha$ is the \textbf{learning rate} and $\alpha \geqslant   0$.
    \item It is importante to make the simultaneous update of $\theta_0$ and $\theta_1$.
    \item "Batch" Gradient descent means that each step of gradient descent uses all the training examples, i.e., all the $m$ examples.
  \end{itemize}
  For Linear Regression:
  \begin{itemize}
    \item $J(\theta_0, \theta_1)$ will always be a Convex Function, i.e., a "bowl-shaped function"
    \item This implies that it does not have any local optima, only the global one.
  \end{itemize}

  \section{Prediction and Inference}

  \begin{itemize}
    \item Prediction: you use variables to predict \textbf{values}
    \item Inference: you use variables to understand behaviors, find structures, etc
  \end{itemize}

  \subsection*{Prediction accuracy vs. Model interpretability}

  Usually, when you want to \textit{predict}, you don't care very much about \textbf{interpretability}, because what matters is the accuracy and not how variables are correlated, and then you can have more \textbf{flexibility}. On the other hand, when you want to make an \textit{inference}, you also want to \textbf{interpret} the algorithm's outcomes, therefore you will end up using less \textbf{flexible} methods.

  \begin{quotation}
    \textit{"In general, as the flexibility of a method increases, its interpretability decreases."} - ISLR-Gareth \cite{gareth}.
  \end{quotation}

\begin{figure}[h]
  \includegraphics[width=0.75\textwidth]{trade-off_garneth}
  \caption{How the learning algorithms are classified in terms of flexibility vs. interpretability, from ISLR-Gareth \cite{gareth}.}
  \label{}
\end{figure}

\section{Multivariate Linear Regression}

Let
\begin{equation}
  \mathbf{x}=
  \begin{bmatrix}
    x_0 \\ x_1 \\ x_2 \\ \vdots \\ x_n
  \end{bmatrix} \in \mathbb{R}^{n+1}
  \quad \text{and} \quad
  \theta=
  \begin{bmatrix}
    x_0 \\ x_1 \\ x_2 \\ \vdots \\ x_n
  \end{bmatrix} \in \mathbb{R}^{n+1}
\end{equation}

then the Multivariate Linear Regression is defined by
\begin{equation}
  h_\theta(x)=\theta_0 x_0 + \theta_1 x_1 + \dots + \theta_n x_n = \theta^T \mathbf{x}
\end{equation}

\subsection*{Gradient descent}
In the case $n\geq 1$ case, you should repeat:
\begin{equation}
  \theta_j := \theta_j - \alpha\frac{1}{m}\sum_{i=1}^m (h_\theta (x^{(i)})-y^{(i)})x_{j}^{(i)}
\end{equation}

There are some tricks to help the Gradient Descent converge more quickly:
\begin{itemize}
  \item Feature scaling
    \subitem get every feature into approximately a $-1 \leq x \leq 1$ range
    \subitem mean normalization: $\frac{x_i-\mu_i}{s_i}$ where $\mu_i$ is the mean of the $x_i$ and $s_i$ is the range.
  \item Debugging: your $J(\theta)$ should descrease at every iteration. If it's increasing, or if it shows an "weird" behavior (like descreasing and increasing over and over) maybe you should use a smaller $\alpha$. But if $alpha$ is too small, gradient descent can be slow to converge.
\end{itemize}

\subsection*{Normal equation}
It is a method to solve for $\theta$ "analytically". In this case, you don't need to use feature scaling.

\begin{equation}
  \theta=(X^T X)^{-1}X^T y
\end{equation}

Gradient descent
\begin{itemize}
  \item Need choose $\alpha$
  \item Needs many itrations
  \item Works well even when $n$ is large
\end{itemize}

Normal equation
\begin{itemize}
  \item No need to chosse $\alpha$
  \item Don't need to iterate
  \item Need to compute $(X^TX)^{-1}$
  \item Slow if $n$ is very large (usually for $n\geq 10000$)
\end{itemize}

$X^TX$ could be non-invertible if
\begin{itemize}
  \item The features are linear dependente
  \item You have too many features. You should delete some features or use regulatization.
\end{itemize}

\clearpage
\begin{thebibliography}{99}
  \bibitem{gareth}
  Gareth James, Daniela Witten, Trevor Hastie and Robert Tibshirani,
  \textit{Introduction to Statistical Learning with Applications in R},
  Springer,
  2017.

\end{thebibliography}

\end{document}
